\documentclass[a4paper,12pt, fontset=none, hyperref]{ctexbook}
\usepackage{ctex}
%\ctexset{fontset=ubuntu}
%fonts-arphic-uming
%fonts-arphic-ukai
%fonts-wqy-zenhei
\setCJKmainfont{Source Han Serif TC}
\setCJKsansfont{Source Han Sans TC}

\usepackage{zhlipsum}

\title{Tests of traditional Chinese characters with ctex in fontset=ubuntu}

\begin{document}

\maketitle
\tableofcontents

	\chapter{zh-hant HK vs TW}
	
		This chapter tests some traditional Chinese characters in different regions. The characters have the same meaning but are written differently. They are, however, all traditional Chinese.
	
		The followings are HK version.
		
		歎 爲 着 粧 牀 裏 綫 麪 鈎 衹 羣 醖 衞 才 峯 污
		
		The followings are TW version.
		  
		嘆 為 著 妝 床 裡 線 麵 鉤 只 群 醞 衛 纔 峰 汙
		
		The followings are simplified version.
		
		叹 为 着 妆 床 里 线 面 钩 只 群 酝 卫 才 峰 污

	\chapter{zh-hant HK manual, Shǔ Dào Nán by Lǐ Bái}
	
		噫吁嚱,危乎高哉!蜀道之難,難於上青天!
		蠶叢及魚鳧,開國何茫然!
		爾來四萬八千歲,不與秦塞通人煙。
		西當太白有鳥道,可以橫絕峨眉巔。
		地崩山摧壯士死,然後天梯石棧相鈎連。
		上有六龍回日之高標,下有衝波逆折之回川。
		黃鶴之飛尚不得過,猿猱欲度愁攀援。
		青泥何盤盤,百步九折縈巖巒。
		捫參歷井仰脅息,以手撫膺坐長歎。
		問君西遊何時還?畏途巉巖不可攀。
		但見悲鳥號古木,雄飛雌從繞林間。
		又聞子規啼夜月,愁空山。
		蜀道之難,難於上青天,使人聽此凋朱顏!
		連峯去天不盈尺,枯松倒掛倚絕壁。
		飛湍瀑流爭喧豗,砯崖轉石萬壑雷。
		其險也如此,嗟爾遠道之人胡為乎來哉!
		劍閣崢嶸而崔嵬,一夫當關,萬夫莫開。
		所守或匪親,化為狼與豺。
		朝避猛虎,夕避長蛇。磨牙吮血,殺人如麻。
		錦城雖云樂,不如早還家。
		蜀道之難,難於上青天,側身西望長咨嗟!

	\chapter{zh-hant TW manual, Luò Shén Fù by Cáo Zhí}
	
		黃初三年,余朝京師,還濟洛川。古人有言,斯水之神,名曰宓妃。感宋玉對楚王神女之事,遂作斯賦。其辭曰:
		
		余從京域,言歸東藩。
		背伊闕,越轘轅,經通谷,陵景山。
		日既西傾,車殆馬煩。
		爾迺稅駕乎蘅皋,秣駟乎芝田,容與乎陽林,流眄乎洛川。
		於是精移神駭,忽焉思散,俯則未察,仰以殊觀,覩一麗人,于巖之畔。
		迺援御者而告之曰:
		爾有覿於彼者乎?彼何人斯?若此之豔也?
		御者對曰:
		臣聞河洛之神,名曰宓妃。然則君王所見,無迺是乎?其狀若何?臣願聞之。
		余告之曰:
		其形也,
		翩若驚鴻,婉若游龍。
		榮曜秋菊,華茂春松。
		髣彿兮若輕雲之蔽月,飄颻兮若流風之迴雪。
		遠而望之,皎若太陽升朝霞;
		迫而察之,灼若芙蕖出淥波。
		襛纖得衷,脩短合度。
		肩若削成,腰如約素。
		延頸秀項,皓質呈露。
		芳澤無加,鉛華弗御。
		雲髻峨峨,脩眉聯娟。
		丹脣外朗,皓齒內鮮。
		明眸善睞,靨輔承權。
		瑰姿豔逸,儀靜體閑。
		柔情綽態,媚於語言。
		奇服曠世,骨像應圖。
		披羅衣之璀粲兮,珥瑤碧之華琚。
		戴金翠之首飾,綴明珠以耀軀。
		踐遠遊之文履,曳霧綃之輕裾。
		微幽蘭之芳藹兮,步踟躕於山隅。
		於是忽焉縱體,以遨以嬉。
		左倚采旄,右蔭桂旗。
		攘皓腕於神滸兮,采湍瀨之玄芝。
		余情悅其淑美兮,心振蕩而不怡。
		無良媒以接懽兮,託微波而通辭。
		願誠素之先達兮,解玉佩以要之。
		嗟佳人之信脩兮,羌習禮而明詩。
		抗瓊珶以和予兮,指潛淵而為期。
		執眷眷之款實兮,懼斯靈之我欺。
		感交甫之棄言兮,悵猶豫而狐疑。
		收和顏而靜志兮,申禮防以自持。
		於是洛靈感焉,徙倚彷徨,
		神光離合,乍陰乍陽。
		竦輕軀以鶴立,若將飛而未翔。
		踐椒涂之郁烈,步蘅薄而流芳。
		超長吟以永慕兮,聲哀厲而彌長。
		爾迺眾靈雜遝,命儔嘯侶,
		或戲清流,或翔神渚,
		或采明珠,或拾翠羽。
		從南湘之二妃,攜漢濱之游女。
		歎匏瓜之無匹兮,詠牽牛之獨處。
		揚輕袿之猗靡兮,翳脩袖以延佇。
		體迅飛鳧,飄忽若神,
		陵波微步,羅袜生塵。
		動無常則,若危若安。
		進止難期,若往若還。
		轉眄流精,光潤玉顏。
		含辭未吐,氣若幽蘭。
		華容婀娜,令我忘餐。
		於是屏翳收風,川后靜波。
		馮夷鳴鼓,女媧清歌。
		騰文魚以警乘,鳴玉鸞以偕逝。
		六龍儼其齊首,載雲車之容裔。
		鯨鯢踴而夾轂,水禽翔而為衛。
		於是越北沚,過南岡,紆素領,迴清陽。
		動朱唇以徐言,陳交接之大綱。
		恨人神之道殊兮,怨盛年之莫當。
		抗羅袂以掩涕兮,淚流襟之浪浪。
		悼良會之永絕兮,哀一逝而異鄉。
		無微情以效愛兮,獻江南之明璫。
		雖潛處於太陰,長寄心於君王。
		忽不悟其所舍,悵神宵而蔽光。
		於是背下陵高,足往神留,
		遺情想像,顧望懷愁。
		冀靈體之復形,御輕舟而上溯。
		浮長川而忘反,思綿綿而增慕。
		夜耿耿而不寐,霑繁霜而至曙。
		命僕夫而就駕,吾將歸乎東路。
		攬騑轡以抗策,悵盤桓而不能去。

	\chapter{zh-hans manual, Chū Shī Biǎo by ZhūGé Liàng}
	
		臣亮言:先帝创业未半而中道崩殂,今天下三分,益州疲敝,此诚危急存亡之秋也。然侍卫之臣不懈于内,忠志之士忘身于外者,盖追先帝之殊遇,欲报之于陛下也。
		
		诚宜开张圣听,以光先帝遗德,恢弘志士之气,不宜妄自菲薄,引喻失义,以塞忠谏之路也。
		
		宫中府中,俱为一体,陟罚臧否,不宜异同。若有作奸犯科及为忠善者,宜付有司论其刑赏,以昭陛下平明之治,不宜偏私,使内外异法也。
		
		侍中、侍郎郭攸之、费祎、董允等,此皆良实,志虑忠纯,是以先帝简拔以遗陛下。愚以为宫中之事,事无大小,悉以咨之,然后施行,必能裨补阙漏,有所广益。
		
		将军向宠,性行淑均,晓畅军事,试用之于昔日,先帝称之曰能,是以众议举宠为督。愚以为营中之事,悉以咨之,必能使行阵和睦,优劣得所也。
		
		亲贤臣,远小人,此先汉所以兴隆也;亲小人,远贤臣,此后汉所以倾颓也。先帝在时,每与臣论此事,未尝不叹息痛恨于桓、灵也。侍中、尚书、长史、参军,此悉贞亮死节之臣也,愿陛下亲之信之,则汉室之隆,可计日而待也。
		
		臣本布衣,躬耕于南阳,苟全性命于乱世,不求闻达于诸侯。先帝不以臣卑鄙,猥自枉屈,三顾臣于草庐之中,咨臣以当世之事,由是感激,遂许先帝以驱驰。后值倾覆,受任于败军之际,奉命于危难之间,尔来二十有一年矣。先帝知臣谨慎,故临崩寄臣以大事也。受命以来,夙夜忧叹,恐托付不效,以伤先帝之明,故五月渡泸,深入不毛。今南方已定,兵甲已足,当奖率三军,北定中原,庶竭驽钝,攘除奸凶,兴复汉室,还于旧都,此臣所以报先帝,而忠陛下之职分也。至于斟酌损益,进尽忠言,则攸之、祎、允之任也。
		
		愿陛下托臣以讨贼兴复之效;不效,则治臣之罪,以告先帝之灵。若无兴德之言,则责攸之、祎、允等之慢,以彰其咎。陛下亦宜自谋,以咨诹善道,察纳雅言。深追先帝遗诏,臣不胜受恩感激。
		
		今当远离,临表涕泣,不知所云。

	\chapter{zh-hant lipsum serif}

		\zhlipsum[1-20][name=trad]

	\chapter{zh-hant lipsum sans}

		\textsf{\zhlipsum[1-20][name=xiangyu]}
	
	\chapter{zh-hant lipsum italic}
	
		\textit{\zhlipsum[1-20][name=nanshanjing]}

	\chapter{zh-hans lipsum serif}

		\zhlipsum[1-20]
		
	\chapter{zh-hans lipsum bold serif}

		\textbf{\zhlipsum[1-20]}

	\chapter{zh-hans lipsum italic}

		\textit{\zhlipsum[1-20]}

\end{document}